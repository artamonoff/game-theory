\documentclass[12pt]{article}

\usepackage[utf8]{inputenc}
\usepackage[T2A]{fontenc}
\usepackage[english, russian]{babel}
\usepackage{amsmath, amsthm, amssymb}
\usepackage{enumerate,hhline, bm}
\usepackage[mathscr]{eucal}
%\usepackage{mathtext}

\usepackage{hyperref}
\hypersetup{unicode=true, final=true, colorlinks=true}


%
% Теоретико-игровые
%
\newcommand{\StrategySet}{{\mathfrak S}}
\newcommand{\Strategy}{s}
\newcommand{\StrategyA}{A}
\newcommand{\StrategyB}{B}
\newcommand{\Agent}{{\mathfrak A}}
\DeclareMathOperator{\Exp}{E}

%
%  Линейная алгебра
%
\DeclareMathOperator{\rank}{rank}
\DeclareMathOperator{\dimension}{dim}
\newcommand{\LinearSpace}{{\mathfrak L}}
\DeclareMathOperator{\trace}{tr} % След матрицы
\newcommand{\Lin}{{\mathfrak H}}

%
%   Числовые
%
\newcommand{\Complex}{{\mathbb C}}
\newcommand{\N}{\mathbb N}
\newcommand{\Z}{{\mathbb Z}}
\newcommand{\Q}{{\mathbb Q}}
\newcommand{\R}{{\mathbb R}}
\newcommand{\semiaxes}{{\mathbb R_+}}
\DeclareMathOperator{\Real}{Re}
\DeclareMathOperator{\Image}{Im}
\DeclareMathOperator{\dist}{dist}

%
%   Вектора
%
\newcommand{\vconst}{{\bm const}}
\newcommand{\vectx}{{\bm x}}
\newcommand{\vecty}{{\bm y}}
\newcommand{\vectz}{{\bm z}}
\newcommand{\vecte}{{\bm e}}
\newcommand{\vectw}{{\bm w}}
\newcommand{\vecth}{{\bm h}}
\newcommand{\vectr}{{\bm r}}
\newcommand{\vectq}{{\bm q}}
\newcommand{\vectf}{{\bm f}}%{\boldsymbol{f}}
\newcommand{\vectu}{{\bm u}}
\newcommand{\vectv}{{\bm v}}
\newcommand{\vectc}{{\bm c}}
\newcommand{\vectQ}{{\bm Q}}
\newcommand{\vectone}{{ 1}}
\newcommand{\vectalpha}{{\bm{\alpha}}}
\newcommand{\vectbeta}{{\bm{\beta}}}
\newcommand{\vectgamma}{{\bm{\gamma}}}
\newcommand{\vectdelta}{{\bm{\delta}}}
\newcommand{\vecteta}{{\bm{\eta}}}
\newcommand{\vectpi}{{\bm{\pi}}}
\newcommand{\vectmu}{{\bm{\mu}}}
\newcommand{\vectlambda}{{\bm{ \lambda}}}
%
%  Матрицы
%
\newcommand{\Id}{I}
\newcommand{\matrixX}{{\bm X}}
\newcommand{\matrixY}{{\bm Y}}
\newcommand{\matrixU}{{\bm U}}
\newcommand{\matrixV}{{\bm V}}
\newcommand{\matrixR}{{\bm R}}
\newcommand{\matrixZ}{{\bm Z}}
\newcommand{\matrixA}{{\bm A}}
\newcommand{\matrixB}{{\bm B}}
\newcommand{\matrixQ}{{\bm Q}}
\newcommand{\matrixH}{{\bm H}}
\newcommand{\matrixM}{{\mathscr M}}
\newcommand{\matrixGamma}{{\bm{\Gamma}}}
\newcommand{\matrixPi}{{\bm{\Pi}}}
\newcommand{\Minor}{{\mathcal M}} % Минор матрицы.

%
%  Мат.анализ
%
\newcommand{\Hamiltonian}{{\mathscr H}} % Гамильтониан
\newcommand{\HamiltonianG}{{\mathfrak H}} % Гамильтониан
\newcommand{\Hessian}{{\mathsf {Hess}}} % Hessian matrix
\newcommand{\BordHessian}{\mathsf{Hess}} % Bordered Hessian
\newcommand{\DomV}{{\mathcal V}}
\newcommand{\DomU}{{\mathcal U}}
\newcommand{\DomD}{{\mathscr D}}
\newcommand{\FuncF}{{\mathscr F}} % Функционал F
\newcommand{\FuncL}{{\mathscr L}} % Функционал L
\newcommand{\Lagrange}{{\mathscr L}}
\newcommand{\LagrangeZ}{{\mathscr Z}}
\DeclareMathOperator{\Domain}{dom} %Область определения
\DeclareMathOperator{\argmax}{argmax}
\DeclareMathOperator{\argmin}{argmin}
\DeclareMathOperator{\grad}{grad}

%
% Теоремы, Примеры etc
%
\theoremstyle{plain}
\newtheorem*{teorema}{Theorem}
\newtheorem*{importante}{Important!}
\newtheorem*{ejemplo}{Example}
\newtheorem*{definicion}{Definition}
\newtheorem*{col}{Collorary}
\newtheorem*{propuesta}{proposition}

\theoremstyle{remark}
\newtheorem*{remark}{Remark}


\theoremstyle{remark}
\newtheorem{exercise}{}[subsection]
\renewcommand{\theexercise}{\textbf{\textnumero \arabic{exercise}}}

%\DeclareMathOperator{\cov}{cov}
%\DeclareMathOperator{\corr}{corr}
%\DeclareMathOperator{\Var}{Var}

%\topmargin=-2cm%-1.5cm

%\addtolength{\textheight}{3cm}

%\oddsidemargin=-0.1cm

%\addtolength{\textwidth}{1.8cm}


\title{Problems in Game Theory}
\author{Artamonov N.V.}
%\date{Spring 2014}

%\title{Problems for Exam Preparation in the Course
%<<Methods of Optimal Solutions>>}\author{\copyright Artamonov N.V., Department of EMMAE}

\begin{document}

\maketitle

%\markright{}
\tableofcontents

\section{Zero-Sum Games}

\cite{Asoke}: pp. 364 -- 366

\noindent\cite{Eichhorn} p. 207

\noindent\cite{Peters} pp. 33 -- 35


\begin{exercise}%[\textbf{5 points} for each item]
Two players play for money, simultaneously naming one of the numbers 1 or 2,
and then calculating the sum $S$. If $S$ is even, then the first player wins $S$ dollars from the second,
if $S$ is odd, then the second wins $S$ dollars from the first.
\begin{enumerate}
	\item Construct the payoff matrix (utility matrix) for each player.
	Will this game be a zero-sum game? % Justify your answer.
	\item Will there be dominant strategies in this game? % Explain your answer.
	\item Will there be Nash equilibria in pure strategies? % Justify your answer.
	\item Suppose players follow mixed strategies
	\(P^\top=\begin{pmatrix} 0.3 & 0.7\end{pmatrix}\) and \(Q^\top=\begin{pmatrix} 0.25 & 0.75\end{pmatrix}\).
	Compute the expected payoff for each player.
	\item Find the Nash equilibria in mixed strategies.
	% \item Give an interpretation of the equilibrium (Nash) strategies.
	\item Find the expected payoff (utility) for each player
	in the Nash equilibrium in mixed strategies.
\end{enumerate}
\end{exercise}

\begin{exercise}
Solve the previous problem under the condition that players name one of
the following numbers: 1, 2, or 3.
\end{exercise}

\begin{exercise}
Consider an antagonistic game with the matrix
\[
	\begin{pmatrix}
	-2 & 2 & -1 & 0 & 1 \\
	2 & 3 & 1 & 2 & 2 \\
	3 & -3 & 2 & 4 & 3 \\
	-2 & 1 & -2 & -1 & 0
	\end{pmatrix}
\]
\begin{enumerate}
	\item Find the upper and lower value of the game.
	\item Does an equilibrium in pure strategies exist? Explain your answer.
	\item Can the size of the game's payoff matrix be reduced?
	% What is this method called and what does it consist of?
	% \item Solve this problem using the geometric method for $2\times n$ games.
	\item Find the Nash equilibrium by reducing it to a linear
	programming problem % (solve graphically).
\end{enumerate}
\end{exercise}

\begin{exercise}
Apply the dominance operation to the payoff matrices. Conduct an analysis of the game before and
after the dominance operation. Find the Nash equilibrium and the value of the game.
\begin{align*}
	a)&\;\begin{pmatrix} 2 & 1& 0 & -1 & 4 \\ 3 & 4 & 1 & 1 & 4 \\
	-1 & 0 & 2 & -4 & 1 \\ 2 & 2 & 3 & 1 & 3 \\ 4 & 5 & 3 & 1 & 2 \end{pmatrix} &
	b)&\; \begin{pmatrix} 1 & 4 & 0 & -3 & 2 \\ 3 & 3 & 2 & 4  & 1 \\ 2 & 5 & 1 & 2 & 3 \\ 
	3 & 4 & -1 & 0 & 2 \\ 2 & 2 & 1 & 1 & 0 \end{pmatrix} \\
	c)&\; \begin{pmatrix} 4 & 3 & 3 & 4 & 4 \\ 3 & -1 & -5 & 1 & 5 \\ 8 & 2 & -6 & 0 & -5 \\ 
	2 & 0 & 1 & 4 & 5 \\ 2 & 1 & 3 & 5 & 6 \\ 4 & 4 & 3 & 6 & 5 \end{pmatrix}
\end{align*}
\end{exercise}

\begin{exercise}
Consider an antagonistic game with the matrix
\[
	\begin{pmatrix}
	2 & -1 \\ -2 & 1
	\end{pmatrix}
\]
\begin{enumerate}
	\item Consider the mixed strategies of the players
	\begin{align*}
		P^\top&=\begin{pmatrix} 0.4 & 0.6 \end{pmatrix} &
		Q^\top&=\begin{pmatrix} 0.8 & 0.2 \end{pmatrix}
	\end{align*}
	Find the expected payoffs for each player.
	\item Find the Nash equilibria and the value of the game.
\end{enumerate}
\end{exercise}

\begin{exercise}
Find the Nash equilibrium in mixed strategies and the value of the game in
the zero-sum game
\[
	\begin{pmatrix}
	-20 & 2 & 22 & -15 \\ 20 & -8 & -11 & 0
    \end{pmatrix}
\]
\end{exercise}

\begin{exercise}
For the antagonistic game with the matrix
\begin{align*}
	a)&\;\begin{pmatrix} 4 & 2 & 1 & 5 \\ 2 & 3 & 6 & 3 \end{pmatrix} &
	b)&\; \begin{pmatrix} 2 & 4 \\ 0 & 5 \\ 2 & 6 \\ 3 & -4 \\ 1 & 5 \\ 3 & -1\end{pmatrix} &
	c)&\; \begin{pmatrix} -2 & 3 & 4 & 1 & 3 \\ 6 & -5 & 3 & 3 & -1 \end{pmatrix}
\end{align*}
find the Nash equilibrium.
\end{exercise}

\begin{exercise}
The "weapon jamming" game. Side $A$ has three types of weapons
$A_1,A_2,A_3$, and side $B$ has three types of jammers $B_1,B_2,B_3$. The probability of solving
the combat task by side $A$ for different types of weapons and jammers is given by the matrix
\begin{center}
	\begin{tabular}{|c|c|c|c|}\hline
	& $B_1$ & $B_2$ & $B_3$ \\ \hline
	$A_1$ & 0.8 & 0.2 & 0.4 \\ \hline
	$A_2$ & 0.4 & 0.5 & 0.6 \\ \hline
	$A_3$ & 0.1 & 0.7 & 0.3 \\ \hline
	\end{tabular}
\end{center}
Side $A$ aims to solve the combat task, side $B$ aims to prevent it.
\begin{itemize}
	\item Find the upper and lower value of the game. Will there be Nash equilibria
	in pure strategies in this game?
\end{itemize}
For convenience, multiply the matrix by 10.
\begin{itemize}
	\item Write down the pair of dual problems for finding the equilibrium in mixed strategies.
\end{itemize}
Let the optimal solutions of the dual linear programming problems be known:
for player $A$
\begin{align*}
	x_1&=\frac{1}{32} & x_2&=\frac{3}{16} & x_3&=0
\end{align*}
for player $B$
\begin{align*}
	y_1&=\frac{3}{32} & y_2&=\frac{4}{32} & y_3&=0
\end{align*}
\begin{itemize}
	\item Find the optimal strategies for each player and the value of the game.
\end{itemize}
\end{exercise}

\begin{exercise}
Colonel Blotto commands three units. There are three hills in front of him.
He must decide how many units to send to capture each hill.
His opponent, Count Baloney, also commands three units and must make the same decision.
If on one hill one opponent has a numerical superiority, then he captures that hill.
If not, the hill remains neutral territory.
The payoff for each player is the number of hills captured by him,
minus the number of hills captured by the opponent.
Construct the game matrix and find the Nash equilibrium in this game.
\end{exercise}


\section{Non-Zero-Sum Games}

\begin{exercise}
Consider the payoff matrix of participants A and B in a certain paired tournament,
who adhere to one of two strategies in it
%bimatrix game
\begin{center}
	\begin{tabular}{|c||c|c|}
	\hline
	% after \\: \hline or \cline{col1-col2} \cline{col3-col4} ...
	& $s_{-1}$ & $s_{-2}$  \\ \hline \hline
	$s_1$ & 1, 2 & 4, 3  \\ \hline
	$s_2$ & 3, 4 & 2, 3  \\ %\hline
	\hline
	\end{tabular}
\end{center}
\begin{enumerate}
	\item Consider the mixed strategies of the players
	\begin{align*}
		P^\top&=\begin{pmatrix} 0.7 & 0.3 \end{pmatrix} &
		Q^\top&=\begin{pmatrix} 0.6 & 0.4 \end{pmatrix}
	\end{align*}
	Find the expected payoffs for each player.
	\item Find the Nash equilibria in pure and mixed strategies.
\end{enumerate}
\end{exercise}

\begin{exercise}
Consider the bimatrix game
\begin{center}
	\begin{tabular}{|c||c|c|}
	\hline
	% after \\: \hline or \cline{col1-col2} \cline{col3-col4} ...
	& $s_{-1}$ & $s_{-2}$  \\ \hline \hline
	$s_1$ & 4, 2 & 2, 0  \\ \hline
	$s_2$ & 2, 2 & 3, 5  \\ %\hline
	\hline
	\end{tabular}
\end{center}
\begin{enumerate}
	\item Consider the mixed strategies of the players
	\begin{align*}
		P^\top&=\begin{pmatrix} 0.5 & 0.5 \end{pmatrix} &
		Q^\top&=\begin{pmatrix} 0.2 & 0.8 \end{pmatrix}
	\end{align*}
	Find the expected payoffs for each player.
	\item Find the Nash equilibria in pure and mixed strategies.
\end{enumerate}
\end{exercise}

\begin{exercise}
Consider the bimatrix game
\begin{center}
	\begin{tabular}{|c||c|c|c|}
	\hline
	% after \\: \hline or \cline{col1-col2} \cline{col3-col4} ...
	& $s_{-1}$ & $s_{-2}$  & $s_{-3}$ \\ \hline \hline
	$s_1$ & 4, 1 & 2, 2 & 1, 3  \\ \hline
	$s_2$ & 2, 2 & 3, 5 & 0, 4 \\ %\hline
	\hline
	\end{tabular}
\end{center}
\begin{itemize}
	\item Consider the mixed strategies of the players
	\begin{align*}
		P^\top&=\begin{pmatrix} 0.6 & 0.4 \end{pmatrix} &
		Q^\top&=\begin{pmatrix} 0.2 & 0.3 & 0.5 \end{pmatrix}
	\end{align*}
	Find the expected payoffs for each player.
	\item Find the Nash equilibria in pure and mixed strategies.
\end{itemize}
\end{exercise}

\begin{exercise}
Consider the bimatrix game
\begin{center}
	\begin{tabular}{|c||c|c|}
	\hline
	% after \\: \hline or \cline{col1-col2} \cline{col3-col4} ...
	& $s_{-1}$ & $s_{-2}$  \\ \hline \hline
	$s_1$ & 4, 2 & 2, 0  \\ \hline
	$s_2$ & 2, 2 & 3, 5  \\ \hline
	$s_3$ & 3, 1 & 2, 3 \\ 
	\hline
	\end{tabular}
\end{center}
\begin{enumerate}
	\item Consider the mixed strategies of the players
	\begin{align*}
		P^\top&=\begin{pmatrix} 0.4 & 0.4 & 0.2 \end{pmatrix} &
		Q^\top&=\begin{pmatrix} 0.3 & 0.7 \end{pmatrix}
	\end{align*}
	Find the expected payoffs for each player.
	\item Find the Nash equilibria in pure and mixed strategies.
\end{enumerate}
\end{exercise}


\begin{exercise}[Cournot Duopoly]
Let $Q_i$ be the output volume, $cQ_i$ --
the costs of firm $i=1,2$. The demand function has the form ($a>c>0$)
\[
	P(Q)=\begin{cases}
	a-Q, & Q\leq a \\
	0, & Q>a
	\end{cases}
\]
The firm's revenue is determined by the equality $(P(Q_1+Q_2)-c)Q_i$.

Each firm has two options: "small-scale production"
$Q^l=(a-c)/4$ and "large-scale production" $Q^h=(a-c)/3$.
\begin{enumerate}
	\item Write the bimatrix game in normal form.
	\item Find the Nash equilibria in pure and mixed
	strategies.
\end{enumerate}
\end{exercise}

\begin{exercise}[Bertrand Duopoly]
Suppose there are two competing firms $A$ and $B$ in the mineral water market.
The fixed costs of each are 300
(regardless of sales volume). Each firm
must choose either a "high" price for its product $P_h=1$,
or a "low" price $P_l=0.5$ (price per bottle). At the "high" price,
1000 bottles can be sold on the market, at the "low" price -- 2000 bottles.
If companies choose the same price, they split sales volumes equally.
If companies choose different prices, the market is completely captured by the company
with the lower price (the other sells nothing).
\begin{enumerate}
	\item Construct the payoff matrix (utility matrix) for each player.
	Will this game be a zero-sum game? % Justify your answer.
	\item Will there be dominant strategies in this game? % Explain your answer.
	\item Will there be Nash equilibria in pure strategies? % Justify your answer.
	\item Find the Nash equilibria in mixed strategies.
	\item Give an interpretation of the equilibrium (Nash) strategies.
	\item Find the expected payoff (utility) for each player
	in the Nash equilibrium in mixed strategies.
\end{enumerate}
\end{exercise}

\begin{exercise}[Bertrand Duopoly]
Suppose there are two competing firms $A$ and $B$ in the mineral water market.
The fixed costs of each are \$5000
(regardless of sales volume). Each firm
must choose either a "high" price for its product $P_h=\$2$, or a "low" price $P_l=\$1$
(price per bottle). Then:
\begin{enumerate}
	\item at the "high" price, 5000 bottles can be sold on the market,
	\item at the "low" price, 10000 bottles can be sold on the market,
	\item if companies choose the same price, they split sales volumes equally,
	\item if companies choose different prices, the market is completely captured by the company
	with the lower price (the other sells nothing).
\end{enumerate}
Construct the game matrix. Will this be a zero-sum game? Find the Nash equilibria
and the value of the game.
\end{exercise}

\begin{exercise}
Consider the bimatrix game
\begin{center}
	\begin{tabular}{|c||c|c|c|c|}
	 \hline
	% after \\: \hline or \cline{col1-col2} \cline{col3-col4} ...
	& $s_{-1}$ & $s_{-2}$ & $s_{-3}$ & $s_{-4}$\\ \hline \hline
	$s_1$ & 2, 2 & 1, 1 & 1, 3 & 2, 1\\ \hline
	$s_2$ & 2, 4 & 1, 3 & 1, 2 & 4, 2\\ \hline
	$s_3$ & 3, 2 & 2, 3 & 2, 5 & 3, 1 \\ \hline
	$s_4$ & 4, 1 & 1, 4 & 3, 1 & 2, 2 \\
	\hline
	\end{tabular}
\end{center}
Perform the procedure of iterated elimination of dominated strategies.
Will there be Nash and Pareto equilibria in pure strategies in this game?
\end{exercise}

\begin{exercise}
Two airline passengers, traveling on the same flight, lost their suitcases.
The airline is ready to compensate each passenger for the damage.
To determine the amount of compensation, each passenger is asked to report
how much they value the contents of their suitcase.
Each passenger can name an amount not less than \$90 and not more than \$100.
The compensation conditions are as follows: if both report the same amount,
then each will receive that amount as compensation.
If the damage claimed by one passenger is less
than the damage claimed by the other passenger, then each passenger will receive compensation
equal to the smaller of the claimed amounts. In this case, the one who claimed the smaller amount
receives an additional \$2, and the one who claimed the larger amount loses an additional two dollars.
Construct the game matrix and find the Nash equilibrium in this game.
\end{exercise}

\begin{exercise}
Winnie the Pooh and Piglet decide who they will visit --- Pooh
(strategy 1), Piglet (strategy 2), or Rabbit (strategy 3).
Friends can go together or separately. From Pooh's point
of view, the pleasure from visiting is assessed as follows: 1, 2
and 4 respectively. From Piglet's point of view, the pleasures are: 3,
2 and 2 respectively. The presence of a companion (a joint visit)
adds one unit of pleasure to each. Find
the equilibrium in this game.
\end{exercise}

\begin{exercise}
Two firms share a market. Each firm has three strategies --- set a
low price, medium price, and high price $p_1=2$, $p_2=4$,
$p_3=7$. Assume that market demand is 100 units
of product; if prices are equal, it is split equally between firms;
if one firm sets a high price and the other a medium price, then demand
is 30 and 70 units respectively. For medium and low
prices, demand is also split 30 and 70, and for high and low
it is 10 and 90 units respectively. Find the
Nash equilibrium.
\end{exercise}

\begin{exercise}
Two radio stations share a market by choosing from two broadcast formats: news and music.
The target audience for each format is 38% and 62% respectively. If radio stations
choose the same broadcast format, they split the target audience in half.
If radio stations choose different broadcast formats, then each ``captures''
the entire target audience of its chosen broadcast format. The radio station's payoff is its audience share.
\begin{enumerate}
	\item Write the game in normal form.
	Will this game be a zero-sum game? % Justify your answer.
	\item Find the Nash equilibria in pure and mixed strategies
	and the players' payoffs at equilibrium.
\end{enumerate}
\end{exercise}

\begin{thebibliography}{99}

\bibitem{Asoke} Asoke Kumar Bhunia, Laxminarayan Sahoo, Ali Akbar Shaikh,
<<Advanced Optimization and Operations Research>>

\bibitem{Eichhorn} Wolfgang Eichhorn, Winfried Gleißner,
<<Mathematics and Methodology for Economics>>

\bibitem{Webb} James N.Webb <<Game Theory Decisions, Interaction and Evolution>>

\bibitem{Hayashi} Takashi Hayashi, <<Microeconomic Theory for the Social Sciences>>

\bibitem{Peters} Hans Peters, <Game Theory A Multi-Leveled Approach>

\end{thebibliography}

% \subsection{Bayesian Games}

% ... (The content of the Bayesian games section would be translated similarly if needed) ...

\end{document}